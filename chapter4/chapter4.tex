\documentclass{article}
% for general math stuff
\usepackage{amsmath}
% for proof
\usepackage{amsthm}
% for number set symbols
\usepackage{amsfonts}
% for links
\usepackage{hyperref}

\author{Arthur Allshire}
\title{A Programmer's Introduction to Mathematics: Chapter 4 Exercise solutions}
\begin{document}
\maketitle

\section*{4.2: Prove De Morgan's Law}
{\it Exercise.} Prove De Morgan's law for sets, which for $A, B \subset X$
states that $(A \cap B)^{C} = A^{C} \cup B^{C}$, and
$(A \cup B)^{C} = A^{C} \cap B^{C}$.

\begin{proof}

\[
\begin{aligned}
    (A \cap B)^{C} &= \{x : x \in A \:\text{and}\: x \in B\}^{C} \\
             &= \{x : x \not\in A \:\text{or}\: x \not\in B\} \\
             &= \{x : x \not\in A\} \cup \{x: x \not\in B\} \\
             &= \{x : x \in A\}^C \cup \{x: x \in B\}^C \\
             &= A^C \cup B^C \\
\\
    (A \cup B)^{C} &= \{x : x \in A \:\text{or}\: x \in B\}^{C} \\
             &= \{x : x \not\in A \:\text{and}\: x \not\in B\} \\
             &= \{x : x \not\in A\} \cap \{x:  x \not\in B\} \\
             &= A^C \cap B^C \\
\end{aligned}
\]

\end{proof}

\section*{4.5: Prove that $\mathbb{N}\times\mathbb{N}$ is countable}

{\it Solution} I had to look this solution up.
\href{https://math.stackexchange.com/questions/91665/proving-mathbbnk-is-countable/91678#91678}{This solution}
is a fairly straightforward method to approach this problem.

\section*{4.6: Union of Countable Sets}
{\it Exercise.} Suppose for each $n \in \mathbb{N}$ we picked a countable set $A_n$. Prove that
the union of all the $A_n$ is countable. Hint: use the previous problem and write the elements of all the
$A_n$ in a grid.

\begin{proof}

    Let $X = A_1 \cup A_2 \cup A_3 \cup ...$. To prove that $X$ is countable we show that there must exist
    a surjection $\mathbb{N} \rightarrow X$. Define a surjective map $g: \mathbb{N} \rightarrow \mathbb{N}^2$.
    From Ex. 4.5, we know such a map must exist. Define $X_{(i, j)}$ as the $j$-th element from the
    $i$-th set which made up X. Now, define $f: \mathbb{N} \rightarrow X$ as $f(x) = X_{g(x)}$,
    which is clearly a surjection (as $g(x)$ is).
    Thus, since the surjection $f: \mathbb{N} \rightarrow X$ exists, $X$ is countable.
\end{proof}

\end{document}

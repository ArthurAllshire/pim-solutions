\documentclass{article}
% for general math stuff
\usepackage{amsmath}
% for proof
\usepackage{amsthm}
% for number set symbols
\usepackage{amsfonts}
% for links
\usepackage{hyperref}
% for declaring delimiter
\usepackage{mathtools}

\DeclarePairedDelimiter\abs{\lvert}{\rvert}%

\author{Arthur Allshire}
\title{A Programmer's Introduction to Mathematics: Chapter 8 Exercise solutions}

\theoremstyle{remark}
\newtheorem*{exercise}{Exercise}

\begin{document}
\maketitle

\section*{8.2.1: Linearity of derivative}
\begin{exercise}
    Prove Theorem 8.9 that the map $f \mapsto f'$ is linear.
\end{exercise}

\begin{proof}
Using the notation in Theorem 8.9, we know that

\[
\begin{aligned}
    D(f+g)(c) &= \lim_{x \to c} \frac{f(x)+g(x) - (f(c) + g(c))}{x-c} \\
              &= \lim_{x \to c} (\frac{f(x)-f(c)}{x-c} + \frac{f(x)-f(c)}{x-c}) \\
    D(f)+D(g) &= (\lim_{x \to c} \frac{f(x)-f(c)}{x-c}) + (\lim_{x \to c} \frac{g(x)-g(c)}{x-c}) \\
    D(kf)(c) &= \lim_{x \to c} \frac{kf(x) - kf(c)}{x-c} \text{, where } k \in \mathbb{R} \\
             &= \lim_{x \to c} k \frac{f(x)-f(c)}{x-c} \\
\end{aligned}
\]

The problem thus boils down to showing that the limit of a sum is the same as the sum of limits,
and that multiplication by a constant inside and outside a limit is equivalent.

To prove the first part, we go back to the definitions of convergance and limits. Let
$x_1, x_2, x_3, ...$ be any sequence converging on $c \in \mathbb{R}$.
Let $f(x_n) \rightarrow L$ and $g(x_n) \rightarrow M$ for any sequence $x_n \rightarrow c$.
By the definition of convergence,
$\forall \delta > 0, \exists k \in \mathbb{N}$ such that both
$\abs{f(x_n) - L} < \delta$ and $\abs{g(x_n) - M} < \delta$ for each $n>k$.
We now construct the series $f(x_1)+g(x_1), f(x_2)+g(x_2), f(x_3)+g(x_3), ...$.
Adding the two previous equations,

\[
\begin{aligned}
    \abs{f(x_m)-L} + \abs{g(x_m)-M} &< 2\delta \\
    \abs{(f(x_m)+g(x_m))-(L+M))} &< \epsilon \text{, where } \epsilon = 2\delta \\
\end{aligned}
\]
Since $\delta$ may be any real number greater than 0, so must $\epsilon$, and hence
$f(x_n)+g(x_n) \rightarrow L+M$. So,
\[
\begin{aligned}
    \lim_{x \to c} f(x) + g(x) &= L+M \\
    &= \lim_{x \to c} f(x) + \lim_{x \to c} g(x) \\
\end{aligned}
\]
Similarly, let $x_n$ be a sequence such that $x_n \rightarrow c$, and
$f(x)$ be a function such that the series $f(x_n) \rightarrow L$,
for every series $x_n$. Thus there is a value $k \in \mathbb{N}$
such that $\abs{f(x_n)-L} < \delta, \forall \delta > 0$ for each
$n > k$.
Let $af(x_1), af(x_2), ...$ be another sequence. Multiplying the
previous expression by some $a \in \mathbb{R}$,
\[
\begin{aligned}
    a\abs{f(x_n)-L} &< a\delta \\
    \abs{af(x_n)-aL} &< \epsilon \text{, where } \epsilon = a\delta \\
\end{aligned}
\]
Since $\delta$ may be any real number greater than 0, so must $\epsilon$,
and hence the limit of $af(x_n)$ is $aL$. This implies that
$\lim_{x \to c} af(x) = a\lim_{x \to c} f(x)$, as desired,
thus completing the proof.

\end{proof}

\section*{8.2.2: Product of limits}
\begin{exercise}
    Using the definition of the limit of a function, prove that:
    \[
        \lim_{x \to a}[f(x)g(x)] = (\lim_{x \to a}f(x)(\lim_{x \to a} g(x))
    \]
\end{exercise}

\begin{proof}
Let $a_1, a_2,...$ be a series converging on L, and $b_1, b_2,...$ be a series converging on M. For every threshold $\epsilon > 0$ there is a $k \in \mathbb{N}$ such that all the $a_n$ after $a_k$, and all of the $b_n$ after $b_k$ are within $\epsilon$ of $L$

We now construct the series $a_1b_1,a_2b_2,...$. To prove it converges on $LM$ we must show that $\abs{a_nb_n - LM} < \delta$ for all $\delta > 0$ Clearly now for all $n>k$, the term $a_nb_n$ is within $(L+\epsilon)(M+\epsilon)$ of $LM$. Letting $\delta = \epsilon^2+\epsilon(L+M)$, we then obtain (by the expansion of the previous binomial) $\abs{a_nb_n - LM} < \delta$. Since $\epsilon$ may be any real number greater than 0, so must $\delta$ and hence the series converges on $LM$.

If $\lim_{x \to c} f(x) = L$ and $\lim_{x \to c} g(x) = M$, then for all series $x_n \rightarrow c$, $f(x_n) \rightarrow L$ and $g(x_n) \rightarrow M$, and thus using the above $f(x_n)g(x_n) \rightarrow LM$, and so
\[
\begin{aligned}
\lim_{x \to c} f(x)g(x) = (\lim_{x \to c} f(x))(\lim_{x \to c} g(x))
\end{aligned}
\]
\end{proof}

\end{document}
